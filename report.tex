\documentclass[runningheads]{llncs}

\usepackage{xcolor}
\usepackage{listings}
\usepackage{hyperref}
\usepackage{graphicx}
\usepackage{mdframed}

\definecolor{light-gray}{gray}{0.95}
\hypersetup{
    colorlinks,
    citecolor=black,
    filecolor=black,
    linkcolor=black,
    urlcolor=black
}
\title{COMP472 - Reports}
\author{Thomas Backs\inst{1} \and Marco Tropiano\inst{2}\and Earl Aromin Steven\inst{3}}
\authorrunning{T. Backs, M. Tropiano \& E. Armoin}
\institute{27554524 - thomasbacks@gmail.com \and 26789331 - tropiano.m@gmail.com \and 40004997 - earlaromin@gmail.com} 

\begin{document}

\maketitle

\newpage

\section{Introduction}
This project is completed in the context of COMP472, Artifical Intelligence for Fall semester in 2019.
The objective of this project is to make a simple game called X-Rudder using AI and different decision-
making algorithm, this report explains in detail what have we done and the reason behind our choice of
AI algorithm as well. The team member of this project are Thomas Backs, Marco Tropiano, and Earl Aromin
Steven. In team of three (3), we made a game that can be played in either Player vs Player (PvP) or
Player vs Comuter (PvC). The game is in Command Line Interface (CLI), because the focus of this project
is the AI of the game rather than GUI. \textcolor{red}{\textit{Add more information such as Algorithm chosen}}

\subsection{Technical details}
The goal of this project is to make an interesting game using Artifical Inteligence, in order to run this
the prerequisites to runs this are \textbf{Python3} and \textbf{numpy package}. We use Python3 because we
wanted to have recent stable version of it since Python 2.7 will be deprecate past 2020. The numpy package
is required for our 2D board, this package made it easier to handle it.\newline

To run it simply type this following command line \textbf{python3 x-rudder.py}. It will enter the game
loop, from there you have the option to run it in PvP mode or PvC mode.

% 1 to 1 and half plage for the section below.
\section{Heuristic}
...
\subsection{Description}
...
\subsection{Justification}
...

% 1 page for the section below.
\section{Tournament or When we player against it}
...
\subsection{Analysis}
...
% why our heuristic make good/bad decision based on test.
\subsection{Result}
...

%1/2 to 1 page for section below.
\section{Issues encountered}
...
%1/2 page
\section{Responsibilities}
\subsection{Earl Aromin Steven}
\subsection{Marco Tropiano}
\subsection{Thomas Backs}
% Below is just the template for the python code we add. 
\begin{mdframed}[backgroundcolor=light-gray,roundcorner=10pt,leftmargin=1, rightmargin=1, innerleftmargin=15, innertopmargin=15,innerbottommargin=15, outerlinewidth=1, linecolor=light-gray]
    \begin{lstlisting}[language=Python]
        import config
        def main() {
            print('test')
            f = 145
            return f
        }
        
    \end{lstlisting}
\end{mdframed}

    
\end{document}
